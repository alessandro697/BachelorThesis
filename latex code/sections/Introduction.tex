%In this work we address the problem of the motion of a complex bistable diatomic biological molecule moving in an external corrugated potential, using the simplified model introduced in Ref.\cite{Cavallini}. The motion is studied in one dimension, leading to an especially simple, yet instructive, formulation of the problem, whose outcomes are expected to provide useful insights for the general three-dimensional motion.
This thesis can be considered as a prosecution of Ref.~\cite{Cavallini}. It focuses on the oscillations occurring in a regime of steady sliding of a diatomic molecule, pulled over a periodic corrugation by a constant force. The constituents of the molecule are bounded together by a bistable molecular potential. The motion is studied in one dimension, leading to an especially simple, yet instructive, formulation of the problem, whose outcomes are expected to provide useful insights for the general three-dimensional motion. In the present work we stick to zero temperature. Refs.~\cite{Fusco} and~\cite{Fasolino} deal with the effects of a finite temperature, but do not adopt a bistable internal potential. 

The model employed is described in Sec.~\ref{model:sec}. In Sec.~\ref{oscillations:sec} we provide an explication for the peculiar oscillations of the internal motion of the molecule, that spans from a local maximum to a local minimum of the molecular binding potential, already identified in Ref.~\cite{Cavallini}. In Sec.~\ref{resonances:sec} we discuss the applicability of a harmonic approximation for the oscillations in the molecular potential, and evaluate the extra friction that the molecule experiences when pulled over the periodic corrugation during steady advancement. While in Secs.~\ref{oscillations:sec} and~\ref{resonances:sec} we study the system in an overdamped regime, in Sec.~\ref{underdamped:sec} we give a brief overview of the oscillations occurring in the underdamped case. In Sec.~\ref{different:sec} we study the effects of a difference in the damping factor of the different constituents of the molecule on the internal oscillatory dynamics. In Sec.~\ref{conclusion:sec} we discuss the obtained results and some futures development of the problem. Finally, in Appendix~\ref{appendix:sec} is described the utilised numerical integration method.