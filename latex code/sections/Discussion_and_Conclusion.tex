In this thesis work we study the behaviour of a bistable molecule driven across a periodic corrugated substrate by a constant external force, focusing on the steady sliding motion after the initial time transient. In particular we study a peculiar kind of oscillation of the internal motion of the molecule, spanning from the maximum of the internal potential at $R_1$ to the minimum at $R_2$, as reported in Figs.~\ref{Fig:11_a_R_Forzante} and~\ref{Fig:11_b_R_Forzante}.

In Sec.~\ref{oscillations:sec} we verify that the frequency of the internal motion of the molecule is determined by the mean center-of-mass velocity trough the washboard frequency $\nu_\text{cm} = \Bar{v}_\text{cm}/a$. In the equation of motion for the relative coordinate $r$, Eq.~\eqref{eq:rdot}, we identify the term responsible for driving the motion, $F_\text{dri}^\text{eff}$, as defined by Eq.~\eqref{eq:driving}. Focusing on the dependence of this force from the parameters $R_1$ and $R_2$ of the internal potential, we identify a necessary condition for the occurrence of the peculiar kind of oscillations reported in Figs.~\ref{Fig:11_a_R_Forzante} and~\ref{Fig:11_b_R_Forzante}. Since $F_\text{dri}^\text{eff}$ vanishes when $r = na$, where $n$ is an integer number, no such points must be present between $R_1$ and $R_2$. 

In Sec.~\ref{resonances:sec} we conclude that an harmonic approximation for the oscillations around the minimum of the molecular potential at $R_2$ is valid only for large values of the potential barrier $\delta$, when only small-amplitude displacements are allowed. In this latter case, we identify a resonance peak in the amplitude of the oscillation when the washboard frequency approaches the natural oscillation frequency of the minimum of $V_\text{int}$ at $R_2$. We also study the extra friction exerted on the molecule in a state of constant advancement over the corrugated potential, and we identify a peak in the friction corresponding to the peak in the amplitude mentioned above. When the harmonic approximation is applicable, we retrieve the same results of Ref.~\cite{Fusco} for the zero temperature driven dimer. 

In Sec.~\ref{underdamped:sec} we provide a preliminary investigation of the underdamped dynamics. We find that the underdamped system exhibits a larger set of possible oscillations of the internal coordinate $r$. In particular, very wide oscillations with amplitude $A>2R_2$ and oscillations around the origin are permitted. We find that the underdamped system is more sensible than the overdamped one to the initial conditions. This underdamped dynamics is certainly irrelevant for long polymers in solution, but may provide interesting information for the physics of small bistable molecules in vacuum. In this latter case however, quantum-mechanical effects may play a role too: for accounting them, one should investigate the current model within quantum mechanics instead of classical mechanics.

Finally, in Sec.~\ref{different:sec} we study the effect of having two different damping factor for the two particles. We find that a difference in the damping factor effectively modify the internal potential, shifting the minima at zero and at $R_2$.

Future analysis of this bistable system could include the effects of a finite temperature. Allowing for different masses of the particles could also results in a different behaviour of the system. An analysis in two dimensions could also be conducted. This latter situation could be useful to model the sliding of a molecule on a real-life two-dimensional surface. 
